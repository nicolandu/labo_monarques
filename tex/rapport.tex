%filename Rapport-asclepiade
% ltex: language=fr
\documentclass{bio}
\author{Harout CHOULGIAN, Akira LAFOREST et Nicolas LANDUCCI}
\title{Analyse de base de données\\TODO}
\teacher{Louis-Philippe Précourt}
\course{BIO-N02}
\group{2}
\department{Biologie}
\date{4 novembre 2025}

\addbibresource{monarques.bib}

\newcommand{\graphic}[2]{%
    \begin{figure}[H]
        \centering
        \caption{\protect#1}
        \resizebox{\textwidth}{!}{
            \input{#2.tikz}% compile script puts it on TEXPATH
        }
    \end{figure}
}

\begin{document}
\maketitle
\tocpage


\section{Introduction}
    CSR: Complètement spatialement aléatoire (\textit{Completely spatially random})\nocite{gbif.orgGBIFOccurrenceDownload2025}\nocite{GlobalAdministrativeAreas2022}

    \graphic{Occurrences de diverses espèces végétales et de lépidoptères à Montréal depuis 2020}{montreal_map}
    \graphic{Corrélation entre les espèces}{correlation}
    \graphic{Corrélation entre les espèces pour $r \le \SI{1}{\km}$}{correlation_close}
    \graphic{Corrélation normalisée entre les espèces pour $r \ge \SI{50}{\m}$}{correlation_norm}
    \graphic{Corrélation normalisée entre les espèces pour $\SI{50}{\m} \le r \le \SI{1}{\km}$}{correlation_close_norm}
    \graphic{Corrélation normalisée entre les espèces pour $r \ge \SI{1}{\km}$}{correlation_far_norm}
\section{Contexte théorique}

\printbibliography
\end{document}

% Biais possible: les gens qui étudient les monarques vont-ils où les asclépiades sont cartographiées?
