%filename Rapport-asclepiade
% ltex: language=fr
\documentclass{bio}

% Per https://wwwnc.cdc.gov/eid/page/scientific-nomenclature:
% For organisms other than bacteria, fungi, and viruses, scientific names of taxa above the genus level (families, orders, etc.) should be in roman type.
\newcommand{\asclepias}{\textit{Asclepias}}
\newcommand{\lepidoptera}{Lepidoptera}
\newcommand{\plantae}{Plantae}

\newcommand{\asclepiassyriaca}{\textit{Asclepias syriaca}}
\newcommand{\asclepiasincarnata}{\textit{Asclepias incarnata}}
\newcommand{\arctiumlappa}{\textit{Arctium lappa}}
\newcommand{\solidagocanadensis}{\textit{Solidago canadensis}}

\newcommand{\danausplexippus}{\textit{Danaus plexippus}}
\newcommand{\limenitisarchippus}{\textit{Limenitis archippus}}

\author{Harout CHOULGIAN, Akira LAFOREST et Nicolas LANDUCCI}
\title{ANALYSE DE BASE DE DONNÉES\\Analyse de la corrélation spatiale entre les espèces du genre \asclepias\ et le papillon monarque (\danausplexippus) sur l'île de Montréal}
\teacher{Louis-Philippe Précourt}
\course{BIO-N02}
\group{2}
\department{Biologie}
\date{6 novembre 2025}

\newcommand{\graphic}[2]{%
    \begin{figure}[H]
        \centering
        \caption{\protect#1}
        \resizebox{\textwidth}{!}{
            \input{#2.tikz}% compile script puts it on TEXPATH
        }
        \label{fig:#2}
    \end{figure}
}

\newcommand{\kab}{K_{AB}(r)}
\newcommand{\normkab}{K_{AB}^*(r)}

\newcommand{\laxqty}[2]{\qty[parse-numbers=false]{#1}{#2}}

\newcommand{\coords}[2]{%
  \ifdim#1pt<0pt
    \number\numexpr-#1\relax\unit{\degree}\,S%
  \else
    #1\unit{\degree}\,N%
  \fi
  \enspace%
  \ifdim#2pt<0pt
    \number\numexpr-#2\relax\unit{\degree}\,O%
  \else
    #2\unit{\degree}\,E%
  \fi
}

\addbibresource{monarques.bib}

\begin{document}
\maketitle
\tocpage

\section{Introduction}
    Le monarque (\danausplexippus) est un papillon parmi les plus emblématiques et étudiés au monde. Espèce migratrice, le monarque passe ses hivers dans les forêts du sud des États-Unis et du centre du Mexique \parencite{cosewicCOSEWICAssessmentStatus2016}. La larve de \danausplexippus\ se nourrit exclusivement d'asclépiades (genre \asclepias), une plante endémique du Canada et des États-Unis sur laquelle elle pond ses \oe ufs \parencite{lunaMonarchsDanausPlexippus2013}. Cependant, ces dernières, en raison de leur croissance agressive, ont été listées jusqu'en~2014 comme des mauvaises herbes nuisibles. À la lumière de l'importance qu'ont ces espèces dans le cycle de vie de \danausplexippus, le genre a été délistée afin d'essayer de contrer le déclin du monarque \parencite{abassTestingGerminationCommon2025}, dont la population semble avoir diminué de \qty{83}{\percent} entre 1994 et~2015 en termes de surface occupée au Mexique \parencite{cosewicCOSEWICAssessmentStatus2016}.

    L'objectif de ce rapport est d'étudier la corrélation spatiale croisée entre les espèces du genre \asclepias\ et \danausplexippus\ dans l'agglomération de Montréal. Étant donné le regroupement des espaces verts à Montréal, la corrélation attendue pour chacune des espèces est positive. De même, \danausplexippus\ se nourrissant exclusivement d'asclépiades au stade larvaire, il est probable que la corrélation entre les observations de monarques adultes et d'asclépiades soit supérieure à celle entre d'autres paires d'espèces.

    Afin de corréler les espèces, des observations du \textit{Global Biodiversity Information Facility} (GBIF), une plateforme ouverte d'agrégation d'occurrences biologiques, seront utilisées. Les observations de diverses espèces végétales, dont le genre \asclepias, seront corrélées en fonction de la distance avec celles d'espèces de lépidoptères (\lepidoptera), dont \danausplexippus.

\section{Contexte théorique}
    La population du papillon monarque est en déclin, un phénomène largement documenté par la communauté scientifique. À ce sujet, les chercheurs s'accordent sur le fait que ce déclin est lié à la destruction des plantes dont se nourrissent les monarques \parencite{boyleMonarchButterflyMilkweed2019}. En effet, les chenilles de \danausplexippus\ se nourrissent exclusivement de feuilles d'asclépiade, une plante toxique. Les monarques ont développé une résistance au glycoside cardiaque, la toxine présente dans ces feuilles. Cette substance est devenue un mécanisme de défense contre les prédateurs, car elle reste dans l'exosquelette des papillons et rend leur consommation dangereuse pour d'éventuels prédateurs, ce qui est reflété par la coloration aposématique du monarque.

    Depuis plusieurs décennies, la population de \danausplexippus\ a diminué dans les milieux urbains et agricoles, à un point tel qu'il est inscrit au Canada en tant qu'espèce en voie de disparition en vertu de la \textit{\citetitle{LoiEspecesPeril2002}} depuis le 8~décembre~2023 \parencite{LoiEspecesPeril2002}. Cette baisse est majoritairement due à l'utilisation croissante d'herbicides, qui réduit la présence des asclépiades. Or, la disponibilité de ces plantes est cruciale pour le développement des papillons monarques \parencite{u.s.nationalparkserviceMilkweedMonarchs2023}.

    En plus de la diminution des populations de sa plante hôte, l'asclépiade, le monarque est également affecté par d'autres facteurs tant biotiques qu'abiotiques, dont les changements climatiques. En effet, le taux de recrutement de \danausplexippus\ (le taux d'individus arrivant à leur maturité) est négativement affecté par des variations de température: la chaleur ou la sécheresse réduisent l'espérance de vie et le taux de reproduction du monarque, tandis que des conditions trop fraîches ou humides réduisent sa durée de croissance et ralentissent son taux de ponte \parencite{ragabImpactClimateChange2025, richImpactTemperatureReproductive2025}. Ainsi, dans chacun des scénarios qu'ils ont évalués, \citeauthor{ragabImpactClimateChange2025} projettent une réduction à long terme de l'habitat stable de \danausplexippus\ par rapport aux modèles à long terme, tant en considérant des scénarios à fortes qu'à faibles émissions de gaz à effet de serre (GES).

    Puisque de multiples freins affectent le cycle de développement de \danausplexippus, il est important d'évaluer la présence du monarque, notamment en milieu urbain. Dans un tel environnement, les conditions de vie de \danausplexippus\ diffèrent grandement de celles en milieu rural, entre autres sur le plan thermique, la présence de surfaces artificielles en milieu urbain engendrant des îlots de chaleur \parencite{batoolHowLatitudeUrban2024}. De plus, le couvert végétal y est significativement plus épars qu'en milieu rural ou forestier: à Montréal, en~2018, les milieux forestiers et humides représentaient une superficie combinée de \qty{43}{\km\squared} sur un territoire de \qty{625}{\km\squared}, soit environ \qty{7}{\percent} du territoire montréalais, une proportion à la baisse depuis~1994. À la même date, les surfaces artificielles y représentaient \qty{449}{\km\squared}, soit environ \qty{72}{\percent} de la même région, en hausse depuis~1994. En comparaison, pour les régions du Québec méridional dont les données de couverture dans les années~2010 sont disponibles, les milieux forestiers et humides et les surfaces artificielles représentaient respectivement environ \qty{72}{\percent} et \qty{2}{\percent} du territoire \parencite{beauregard-desjardinsComptesTerresQuebec2024}, ce qui illustre le contraste en matière de niche écologique accessible pour \danausplexippus\ entre les milieux urbain et rural.

    \subsection{La corrélation spatiale croisée}
        La corrélation croisée entre deux espèces~$A$ et~$B$ permet de mesurer le caractère attractif ou répulsif de l'interaction entre~$A$ et~$B$. Pour ce faire, il est possible d'utiliser l'\textit{indice de Ripley bivarié}, qui mesure l'aire équivalente~$\kab$ du nombre d'individus moyen de l'espèce~$B$ présents dans un rayon~$r$ d'un individu de l'espèce~$A$, dans une zone d'étude donnée \parencite{rossiterTutorialSpatialPoint2024}. Pour une zone d'étude~$W$ ayant~$N_A$ individus de l'espèce~$A$ et~$N_B$ individus de l'espèce~$B$,
        \begin{equation*}
            \kab = \frac{|W|}{N_A N_B} \sum_{a \in A} \sum_{b \in B} \mathbf{1}\!\bigl(d(a,b) \le r\bigr) \frac{L\bigl(c(a,r)\bigr)}{L\bigl(c(a,r) \cap W\bigr)}
        \end{equation*}
        où~$|W|$ est l'aire de $W$, $d(a,b)$ est la distance entre~$a$ et $b$, $c(a,r)$ est un cercle de rayon~$r$ centré en~$a$, $L(x)$ est la longueur de~$x$, et
        \begin{equation*}
            \mathbf{1}\!(\text{condition}) =
            \begin{cases}
                1 & \text{si la condition est vraie}, \\
                0 & \text{sinon}.
            \end{cases}
        \end{equation*}
        Si un point~$a \in A$ est à une distance inférieure à~$r$ du bord de la zone d'étude~$W$, on applique ici une \textit{correction isotrope} en multipliant le poids $\mathbf{1}\!\bigl(d(a,b) \le r\bigr)$ de chaque paire de points $(a,b)$ par l'inverse de la proportion du périmètre du cercle $c(a,r)$ qui est contenue dans~$W$, afin de mitiger les effets de bord.

        Si la distribution de l'espèce~$B$ est complètement spatialement aléatoire (CSR, \textit{completely spatially random}), et peut donc être décrite par un processus de Poisson ponctuel de densité $\lambda_B = N_B/|W|$, la valeur attendue du nombre d'individus de~$B$, dans un rayon~$r$ d'un individu de l'espèce~$A$, $n_B$, est~$\pi r^2 \lambda_B$. Pour un modèle~CSR, la valeur attendue de~$\kab$ est donc~$\pi r^2$. Par conséquent, à l'échelle d'un rayon~$r$, $\kab > \pi r^2$ est indicatif d'une attraction de l'espèce~$B$ par la présence de l'espèce~$A$ et, \textit{a contrario}, $\kab < \pi r^2$ est indicatif d'une répulsion de l'espèce~$B$ par la présence de l'espèce~$A$.

        Il est également possible de normaliser~$\kab$ par rapport au modèle~CSR, selon $\normkab = \kab/\pi r^2$. Dans un tel cas, $\normkab > 1$ est indicatif d'une attraction de l'espèce~$B$ par la présence de l'espèce~$A$, tandis que $\normkab < 1$ est indicatif d'une répulsion de l'espèce~$B$ par la présence de l'espèce~$A$. En général, à mesure que $r$~augmente, $\kab$~tend vers~$\pi r^2$: en effet, une proportion de plus en plus grande de la zone d'étude est évaluée pour chaque individu de~$A$, et $\normkab$~tend vers~\num{1}.

\section{Méthode}
    Les occurrences du \textit{Global Biodiversity Information Facility} correspondant au règne \plantae\ ou à l'ordre \lepidoptera\ dans l'agglomération de Montréal \parencite{gadmGlobalAdministrativeAreas2022, gbif.orgGBIFOccurrenceDownload2025} ont été téléchargées, et les observations datant de ou d'après 2020 et ayant des coordonnées ont été retenues. Deux groupes d'espèces, $\mathbf{A}$~(plantes) et $\mathbf{B}$~(lépidoptères) ont été formés comme suit:
    \begin{itemize}
        \item $\mathbf{A}$: \asclepiassyriaca, \asclepiasincarnata, \arctiumlappa, \solidagocanadensis;
        \item $\mathbf{B}$: \danausplexippus, \limenitisarchippus.
    \end{itemize}
    \arctiumlappa\ et \solidagocanadensis\ ont été choisis comme contrôles négatifs, n'ayant nominalement pas d'interactions avec \danausplexippus\ et \limenitisarchippus.

    Les coordonnées des points, initialement exprimées en WGS84 (EPSG:4326), ont été projetées en NAD83(SCRS)/MTM fuseau~8 (EPSG:2950) afin de minimiser la distorsion spatiale. Pour chacune des paires d'espèces $(A,B) \in \mathbf{A} \times \mathbf{B}$, l'indice de Ripley bivarié~$\kab$ a été calculé pour une plage de valeurs de~$r$ d'environ \num{0}~à~\qty{8000}{\m}, en considérant l'étendue géographique des occurrences et en appliquant une correction isotrope. Pour chacune des paires, un indice normalisé par rapport au modèle~CSR, $\normkab$, a également été calculé, selon $\normkab = \kab/\pi r^2$.

    Un nuage de points des occurrences par espèce a été produit en ajoutant le contour de l'agglomération de Montréal afin d'obtenir une représentation visuelle de la répartition de \danausplexippus\ à l'intérieur de cette dernière (figure~\ref{fig:montreal_map}). De plus, pour chacune des paires~$(A,B)$, un graphique de~$\kab$ a été produit en tenant compte de toutes les valeurs de~$r$ (figure~\ref{fig:correlation}). Afin de mieux visualiser les corrélations pour des valeurs de~$r$ faibles (interactions à courte portée), un second graphique de~$\kab$ a été produit pour $r \le \qty{1}{\km}$ (figure~\ref{fig:correlation_close}).

    De même, pour chacune des paires~$(A,B)$, un graphique de~$\normkab$ a été produit en tenant compte de toutes les valeurs de~$r$ supérieures à~\qty{50}{\m} (figure~\ref{fig:correlation_norm}). En effet, étant donné que $\normkab$~tend vers~$\infty$ à mesure que $r$~tend vers~\num{0}, écarter les points où $r < \qty{50}{\m}$ permet une meilleure résolution verticale du graphique. Afin de mieux visualiser les corrélations pour des valeurs de~$r$ faibles (interactions à courte portée), un graphique de~$\normkab$ a été produit pour $\qty{50}{\m} \le r \le \qty{1}{\km}$ (figure~\ref{fig:correlation_close_norm}). De même, dans le but de mieux visualiser les corrélations pour des valeurs de~$r$ élevées (interactions à longue portée), un graphique de~$\normkab$ a été produit pour $r \ge \qty{1}{\km}$ en écartant les valeurs de~$\normkab$ élevées associées à des valeurs de~$r$ faibles (figure~\ref{fig:correlation_far_norm}). Enfin, un graphique de~$\kab$ pour chaque paire a été créé en normalisant les corrélations avec la corrélation entre \asclepiassyriaca\ et \danausplexippus, $K_{\text{ref}}$ (figure~\ref{fig:correlation_ref_norm}).

\section{Résultats}
    \graphic{Occurrences de diverses espèces végétales et de lépidoptères sur l'île de Montréal depuis 2020}{montreal_map}
    \graphic{Corrélation entre les espèces}{correlation}
    \graphic{Corrélation entre les espèces pour $r \le \qty{1}{\km}$}{correlation_close}
    \graphic{Corrélation normalisée entre les espèces pour $r \ge \qty{50}{\m}$}{correlation_norm}
    \graphic{Corrélation normalisée entre les espèces pour $\qty{50}{\m} \le r \le \qty{1}{\km}$}{correlation_close_norm}
    \graphic{Corrélation normalisée entre les espèces pour $r \ge \qty{1}{\km}$}{correlation_far_norm}
    \graphic{Corrélation entre les espèces normalisée selon $\text{\asclepiassyriaca} \times \text{\danausplexippus}$}{correlation_ref_norm}

    \subsection{Énoncé des résultats}
        Pour chacune des paires d'espèces $(A,B) \in \mathbf{A} \times \mathbf{B}$, une corrélation positive $\bigl(\normkab > 1\bigr)$ est présente pour toutes les valeurs de~$r$, ce qui est indicatif d'un \textit{clustering} positif à toutes les échelles pour chacune des paires d'espèces. De plus, pour toutes valeurs de~$r$ inférieures à environ~\qty{2,2}{\km}, la corrélation entre \asclepiassyriaca\ et \danausplexippus\ est supérieure à celle de toutes les autres paires d'espèces, la seconde corrélation la plus grande pour les mêmes valeurs approximatives de~$r$, soit celle entre \arctiumlappa\ et \danausplexippus, étant d'approximativement~$0,9 K_{\text{ref}}(r)$.

\section{Discussion}
    Les résultats sont indicatifs d'une corrélation positive entre les individus du genre \asclepias\ et \danausplexippus\ pour toutes les valeurs de~$r$ évaluées (environ \num{0}~à~\qty{8000}{\m}): en tous points dans cette plage, $\normkab > 1$. Il est donc possible de rejeter l'hypothèse nulle d'un processus de Poisson ponctuel (modèle~CSR) selon laquelle la distribution de \danausplexippus\ est spatialement indépendante de la distribution des individus du genre \asclepias, $\kab$ étant nettement supérieur à $\pi r^2$ pour chacune des paires d'espèces $(A,B) \in \mathbf{A} \times \mathbf{B}$.

    Cependant, compte tenu de la trame urbaine dans laquelle vit \danausplexippus\ à Montréal, une corrélation positive a également été observée pour toutes les autres paires d'espèces étudiées. En effet, en milieu urbain, la végétation et les milieux naturels sont généralement regroupés et la trame végétale, fragmentée \parencite{knappProtectedAreasUrban2008, niemelaEcologyUrbanPlanning1999}, une tendance à laquelle Montréal ne fait pas exception \parencite{beauregard-desjardinsComptesTerresQuebec2024}. La corrélation positive pour chacune des paires d'espèces $(A,B)$ peut ainsi être partiellement, voire majoritairement, être le simple résultat de la répartition de la végétalisation sur l'île de Montréal (variable confondante), sans pour autant que la présence de~$B$ à un endroit donné résulte et découle de la présence de~$A$ au même endroit. Ainsi, des espèces n'ayant de relation particulière avec les lépidoptères étudiés, comme \solidagocanadensis\ et \arctiumlappa, sont néanmoins fortement corrélés avec ces derniers du fait de leur présence dans les mêmes espaces verts. D'ailleurs, à la figure~\ref{fig:montreal_map}, qui présente la répartition spatiale des observations d'espèces, on peut voir un regroupement d'espèces diverses et non reliées entre elles aux mêmes endroits, notamment dans le parc du Mont-Royal (\coords{45,5}{-73,6}), la forêt de Senneville (\coords{45,43}{-73,95}) et le parc-nature de la Pointe-aux-Prairies (\coords{45,5}{-73,5}).

    Dans le cas d'\asclepiassyriaca\ et de \danausplexippus, cependant, cette corrélation est significativement supérieure à celle entre les autres paires d'espèces étudiées pour des distances~$r$ inférieures à environ~\qty{2,2}{\km}: pour les paires~$(A,B)$ excluant $(\text{\asclepiassyriaca}, \text{\danausplexippus})$, $K_{\text{ref}}$~est jusqu'à environ $6 \times$~plus élevé que~$\kab$. Cette corrélation plus forte dans le cas du papillon monarque et de l'asclépiade commune semble concorder avec le fait que l'asclépiade est un hôte obligatoire de la chenille de \danausplexippus. En effet, \danausplexippus\ ne pouvant pas se développer sans se nourrir de feuilles d'asclépiade, il était attendu d'observer une corrélation entre ces deux espèces qui soit supérieure à celle entre deux espèces quelconques. Ainsi, les mesures de corrélation semblent confirmer que le caractère d'hôte obligatoire de l'asclépiade, particulièrement dans le cas de l'asclépiade commune (\asclepiassyriaca) qui est nettement plus présente à Montréal qu'\asclepiasincarnata, est reflété dans la répartition spatiale du monarque sur le territoire.

    \subsection{Évaluation de la méthode expérimentale et amélioration de la recherche}
       D'abord, la zone d'étude, correspondant à l'agglomération de Montréal (\approx\ île de Montréal et île Bizard), est partiellement couverte d'eau, le périmètre de l'agglomération englobant des cours d'eau, dont le fleuve Saint-Laurent et la rivière des Prairies. Elle contient également de très nombreuses surfaces artificielles, dont la valeur écologique, bien que globalement faible, varie considérablement. En effet, la flore urbaine, présente dans certains quartiers et arrondissements à des échelles variées, renferme une certaine biodiversité que l'on ne retrouve par exemple pas en milieu industriel. Ces différents facteurs, collectivement, sont la principale cause d'erreur dans l'analyse, puisque l'indice de Ripley bivarié avec une simple correction isotrope ne tient pas compte de cette variété de milieux. Pour atténuer cette source d'erreur, il serait possible de restreindre la zone d'étude à l'union de zones à haute valeur écologique, afin de mieux corriger pour les effets de bord au sein de l'agglomération. Il serait par ailleurs envisageable de déterminer un autre facteur correctif à appliquer à chaque paire de points $(a,b)$ dans le calcul de $\kab$ afin de mieux compenser les effets des variations dans l'aménagement du territoire sur les populations des différentes espèces.

       De même, l'étude, qui est principalement basée sur des observations citoyennes, peut présenter un biais dans les observations de \danausplexippus\ du fait qu'il est largement connu que l'asclépiade est un hôte obligatoire de \danausplexippus. En effet, il est possible qu'un biais d'attente ait poussé des naturalistes amateurs à concentrer davantage d'efforts à recenser des monarques dans des lieux réputés riches en asclépiades que dans des lieux quelconques. Conséquemment, cette source d'erreur a potentiellement entraîné une surestimation de la densité relative de \danausplexippus\ autour des asclépiades, et particulièrement autour d'\asclepiassyriaca, compte tenu de la grande présence de cette dernière comparativement à d'autres espèces du genre \asclepias. Pour la même raison, la densité relative de \danausplexippus\ a probablement été sous-estimée dans des lieux pauvres en asclépiades. Cependant, contrairement à de telles distorsions, les sous-estimations du nombre d'individus par un facteur constant ne posent pas un problème dans le calcul de l'indice de Ripley bivarié, le calcul de l'indice normalisant les densités de population à l'échelle de la zone d'étude. Afin de pallier cet enjeu, il serait possible d'effectuer un échantillonnage plus méthodique par la méthode des quadrats. Cette technique, qui consiste en le décompte du nombre d'individus d'une espèce donnée dans une aire définie et en l'extrapolation de cette densité à une aire plus grande, limiterait les biais d'attente et d'échantillonnage liés à la sélection des aires d'échantillonnage.

\section{Conclusion}
    L'étude avait pour but d'établir une corrélation spatiale entre les populations d'asclépiades (genre \asclepias) et de papillons monarques (\danausplexippus) dans l'agglomération de Montréal. Une corrélation positive a été déterminée, à une multitude d'échelles, tant pour cette paire d'espèces que pour d'autres espèces du règne \plantae\ et de l'ordre \lepidoptera. De même, il appert qu'à l'échelle de Montréal, la corrélation entre l'asclépiade commune et le papillon monarque semble plus forte que d'autres espèces, notamment du fait que l'asclépiade est un hôte obligatoire de \danausplexippus. Ainsi, le but de l'étude a été atteint, et l'hypothèse de départ d'une corrélation positive entre les différentes espèces, \textit{a fortiori} entre le genre \asclepias\ et \danausplexippus, a été confirmée. Afin de mieux caractériser les populations d'asclépiades et de monarques à Montréal dans une analyse comparée avec d'autres espèces, il serait souhaitable de passer d'observations \textit{ad hoc} à un décompte exhaustif des individus sur le territoire.

\printbibliography
\end{document}
