%filename Rapport-asclepiade
% ltex: language=fr
\documentclass{bio}
\author{Harout CHOULGIAN, Akira LAFOREST et Nicolas LANDUCCI}
\title{Analyse de base de données\\TODO}
\teacher{Louis-Philippe Précourt}
\course{BIO-N02}
\group{2}
\department{Biologie}
\date{4 novembre 2025}

\addbibresource{monarques.bib}

\newcommand{\graphic}[2]{%
    \begin{figure}[H]
        \centering
        \caption{\protect#1}
        \resizebox{\textwidth}{!}{
            \input{#2.tikz}% compile script puts it on TEXPATH
        }
    \end{figure}
}

% Per https://wwwnc.cdc.gov/eid/page/scientific-nomenclature:
% For organisms other than bacteria, fungi, and viruses, scientific names of taxa above the genus level (families, orders, etc.) should be in roman type.
\newcommand{\asclepias}{\textit{Asclepias}}
\newcommand{\lepidoptera}{Lepidoptera}
\newcommand{\plantae}{Plantae}

\newcommand{\asclepiassyriaca}{\textit{Asclepias syriaca}}
\newcommand{\asclepiasincarnata}{\textit{Asclepias incarnata}}
\newcommand{\arctiumlappa}{\textit{Arctium lappa}}
\newcommand{\solidagocanadensis}{\textit{Solidago canadensis}}

\newcommand{\danausplexippus}{\textit{Danaus plexippus}}
\newcommand{\limenitisarchippus}{\textit{Limenitis archippus}}

\newcommand{\kab}{K_{AB}(r)}
\newcommand{\normkab}{K_{AB}^*(r)}

\newcommand{\laxqty}[2]{\qty[parse-numbers=false]{#1}{#2}}

\begin{document}
\maketitle
\tocpage


\section{Introduction}
    Le monarque (\danausplexippus) est un papillon parmi les plus connus et étudiés au monde. Espèce migratrice, le monarque passe ses hivers dans les forêts du sud des États-Unis et du centre du Mexique \parencite{cosewicCOSEWICAssessmentStatus2016}. La larve de \danausplexippus\ se nourrit exclusivement d'asclépiades (genre \asclepias), une plante endémique du Canada et aux États-Unis sur laquelle elle pond ses \oe ufs \parencite{lunaMonarchsDanausPlexippus2013}. Cependant, ces dernières, en raison de leur croissance agressive, ont été listées jusqu'en 2014 comme des mauvaises herbes nuisibles; à la lumière de l'importance qu'ont ces espèces dans le cycle de vie de \danausplexippus, la famille a été délistée afin d'essayer de contrer le déclin du monarque \parencite{abassTestingGerminationCommon2025}, dont la population semble avoir diminué de \qty{83}{\percent} entre 1994 et 2015 en termes de surface occupée au Mexique \parencite{cosewicCOSEWICAssessmentStatus2016}.

    L'objectif de ce rapport est d'étudier la corrélation spatiale croisée entre les espèces du genre \asclepias\ et \danausplexippus\ dans l'agglomération de Montréal. Étant donné le regroupement des espaces verts à Montréal, la corrélation attendue pour chacune des espèces est positive. De même, \danausplexippus\ se nourrissant exclusivement d'asclépiades au stade larvaire, il est possible de penser que la corrélation entre les observations de monarques adultes et les asclépiades est supérieure à celle entre d'autres paires d'espèces.

    Afin de corréler les espèces, des observations du \textit{Global Biodiversity Information Facility} (GBIF), une plateforme ouverte d'agrégation d'occurrences biologiques, seront utilisées. Les observations de diverses espèces végétales, dont le genre \asclepias, seront corrélées en fonction de la distance avec celles d'espèces de lépidoptères (\lepidoptera), dont \danausplexippus.

\section{Contexte théorique}

\section{Méthode}
    Les occurrences du \textit{Global Biodiversity Information Facility} correspondant au règne \plantae\ ou à l'ordre \lepidoptera\ dans l'agglomération de Montréal \parencite{gadmGlobalAdministrativeAreas2022, gbif.orgGBIFOccurrenceDownload2025} ont été téléchargées, et les observations datant de ou d'après 2020 et ayant des coordonnées ont été retenues. Deux groupes d'espèces, $A$ (plantes) et $B$ (lépidoptères) ont été formés comme suit:
    \begin{itemize}
        \item $A$: \asclepiassyriaca, \asclepiasincarnata, \arctiumlappa, \solidagocanadensis;
        \item $B$: \danausplexippus, \limenitisarchippus.
    \end{itemize}

    Les coordonnées des points, initialement en WGS84, ont été projetées dans le système NAD83(SCRS)/MTM fuseau 8, afin de minimiser la distorsion. Pour chacune des paires d'espèces $(A,B)$, l'indice de Ripley bivarié $\kab$ a été calculé pour une plage de valeurs $r$ d'environ \num{0}~à~\qty{8000}{\m}, en considérant l'étendue géographique des occurrences. Pour chacune des paires, un indice normalisé par rapport au modèle CSR, $\normkab$, a également été calculé, selon
    \begin{equation*}
        \normkab = \tfrac{\kab}{\pi r^2}.
    \end{equation*}

    CSR: Complètement spatialement aléatoire (\textit{Completely spatially random})\nocite{gbif.orgGBIFOccurrenceDownload2025}\nocite{GlobalAdministrativeAreas2022}

\section{Résultats}
    \graphic{Occurrences de diverses espèces végétales et de lépidoptères sur l'île de Montréal depuis 2020}{montreal_map}
    \graphic{Corrélation entre les espèces}{correlation}
    \graphic{Corrélation entre les espèces pour $r \le \SI{1}{\km}$}{correlation_close}
    \graphic{Corrélation normalisée entre les espèces pour $r \ge \SI{50}{\m}$}{correlation_norm}
    \graphic{Corrélation normalisée entre les espèces pour $\SI{50}{\m} \le r \le \SI{1}{\km}$}{correlation_close_norm}
    \graphic{Corrélation normalisée entre les espèces pour $r \ge \SI{1}{\km}$}{correlation_far_norm}

\section{Discussion}

\section{Conclusion}

\printbibliography
\end{document}

% Biais possible: les gens qui étudient les monarques vont-ils où les asclépiades sont cartographiées?
